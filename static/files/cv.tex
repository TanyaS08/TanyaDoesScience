%!TEX TS-program = xelatex
%!TEX encoding = UTF-8 Unicode
% Awesome CV LaTeX Template for CV/Resume
%
% This template has been downloaded from:
% https://github.com/posquit0/Awesome-CV
%
% Author:
% Claud D. Park <posquit0.bj@gmail.com>
% http://www.posquit0.com
%
%
% Adapted to be an Rmarkdown template by Mitchell O'Hara-Wild
% 23 November 2018
%
% Template license:
% CC BY-SA 4.0 (https://creativecommons.org/licenses/by-sa/4.0/)
%
%-------------------------------------------------------------------------------
% CONFIGURATIONS
%-------------------------------------------------------------------------------
% A4 paper size by default, use 'letterpaper' for US letter
\documentclass[11pt,a4paper,]{awesome-cv}

% Configure page margins with geometry
\usepackage{geometry}
\geometry{left=1.4cm, top=.8cm, right=1.4cm, bottom=1.8cm, footskip=.5cm}


% Specify the location of the included fonts
\fontdir[fonts/]

% Color for highlights
% Awesome Colors: awesome-emerald, awesome-skyblue, awesome-red, awesome-pink, awesome-orange
%                 awesome-nephritis, awesome-concrete, awesome-darknight

\definecolor{awesome}{HTML}{529FA8}

% Colors for text
% Uncomment if you would like to specify your own color
% \definecolor{darktext}{HTML}{414141}
% \definecolor{text}{HTML}{333333}
% \definecolor{graytext}{HTML}{5D5D5D}
% \definecolor{lighttext}{HTML}{999999}

% Set false if you don't want to highlight section with awesome color
\setbool{acvSectionColorHighlight}{true}

% If you would like to change the social information separator from a pipe (|) to something else
\renewcommand{\acvHeaderSocialSep}{\quad\textbar\quad}

\def\endfirstpage{\newpage}

%-------------------------------------------------------------------------------
%	PERSONAL INFORMATION
%	Comment any of the lines below if they are not required
%-------------------------------------------------------------------------------
% Available options: circle|rectangle,edge/noedge,left/right

\photo{LogoNetwork.png}
\name{Tanya}{Strydom}

\position{PhD Candidate}
\address{Département de sciences biologiques, Université de Montréal -
Campus MIL, 1375 Ave.Thérèse-Lavoie-Roux, Montréal, QC H2V 0B3, Canada}

\pronouns{she/they}
\email{\href{mailto:tanya.strydom@umontreal.ca}{\nolinkurl{tanya.strydom@umontreal.ca}}}
\homepage{tanyadoesscience.com}
\orcid{0000-0001-6067-1349}
\github{TanyaS08}
\twitter{TanyaS08}

% \gitlab{gitlab-id}
% \stackoverflow{SO-id}{SO-name}
% \skype{skype-id}
% \reddit{reddit-id}


\usepackage{booktabs}

\providecommand{\tightlist}{%
	\setlength{\itemsep}{0pt}\setlength{\parskip}{0pt}}

%------------------------------------------------------------------------------


\usepackage{fontawesome5}

% Pandoc CSL macros
% definitions for citeproc citations
\NewDocumentCommand\citeproctext{}{}
\NewDocumentCommand\citeproc{mm}{%
  \begingroup\def\citeproctext{#2}\cite{#1}\endgroup}
\makeatletter
 % allow citations to break across lines
 \let\@cite@ofmt\@firstofone
 % avoid brackets around text for \cite:
 \def\@biblabel#1{}
 \def\@cite#1#2{{#1\if@tempswa , #2\fi}}
\makeatother
\newlength{\cslhangindent}
\setlength{\cslhangindent}{1.5em}
\newlength{\csllabelwidth}
\setlength{\csllabelwidth}{3em}
\newenvironment{CSLReferences}[2] % #1 hanging-indent, #2 entry-spacing
 {\begin{list}{}{%
  \setlength{\itemindent}{0pt}
  \setlength{\leftmargin}{0pt}
  \setlength{\parsep}{0pt}
  % turn on hanging indent if param 1 is 1
  \ifodd #1
   \setlength{\leftmargin}{\cslhangindent}
   \setlength{\itemindent}{-1\cslhangindent}
  \fi
  % set entry spacing
  \setlength{\itemsep}{#2\baselineskip}}}
 {\end{list}}
\usepackage{calc}
\newcommand{\CSLBlock}[1]{\hfill\break\parbox[t]{\linewidth}{\strut\ignorespaces#1\strut}}
\newcommand{\CSLLeftMargin}[1]{\parbox[t]{\csllabelwidth}{\strut#1\strut}}
\newcommand{\CSLRightInline}[1]{\parbox[t]{\linewidth - \csllabelwidth}{\strut#1\strut}}
\newcommand{\CSLIndent}[1]{\hspace{\cslhangindent}#1}

\begin{document}

% Print the header with above personal informations
% Give optional argument to change alignment(C: center, L: left, R: right)
\makecvheader

% Print the footer with 3 arguments(<left>, <center>, <right>)
% Leave any of these blank if they are not needed
% 2019-02-14 Chris Umphlett - add flexibility to the document name in footer, rather than have it be static Curriculum Vitae
\makecvfooter
  {January 2024}
    {Tanya Strydom~~~·~~~Curriculum Vitae}
  {\thepage}


%-------------------------------------------------------------------------------
%	CV/RESUME CONTENT
%	Each section is imported separately, open each file in turn to modify content
%------------------------------------------------------------------------------



\section{Research Interests}\label{research-interests}

computational ecology; functional traits; ecological networks; species
interactions; FAIR and open science

\section{Education}\label{education}

\begin{cventries}
    \cventry{Université de Montréal}{Doctor of Philosophy in Biological Sciences}{Montréal, Canada}{2020 - Present}{\begin{cvitems}
\item Advisor: T. Poisot, PhD
\item Thesis: Tools 'O the Times: Understanding the common properties of species interaction networks across space
\end{cvitems}}
    \cventry{Stockholms Universitet}{Master of Science in Ecology and Biodiversity}{Stockholm, Sweden}{2018-20}{\begin{cvitems}
\item Advisor: K. Hylander, PhD
\item Thesis: Declines and increases in northern and southern plant populations after changes in the microclimate
\end{cvitems}}
    \cventry{University of Pretoria}{Bachelor of Science (Honours) in Plant Sciences}{Pretoria, South Africa}{2017}{\begin{cvitems}
\item Advisor: P.C. le Roux, PhD
\item Thesis: Bush encroachment in South Africa’s montane grasslands: the impact of \textit{Leucosidea sericea} on microclimate and vegetation
\end{cvitems}}
    \cventry{University of Pretoria}{Bachelor of Science in Ecology}{Pretoria, South Africa}{2014-16}{}\vspace{-4.0mm}
\end{cventries}

\section{Fellowships and Internships}\label{fellowships-and-internships}

\begin{cventries}
    \cventry{Computational Biodiversity Science and Services  (BIOS\textsuperscript{2}) training program}{BIOS\textsuperscript{2} Fellow}{}{2021 - present}{}\vspace{-4.0mm}
    \cventry{Canadian Institute of Ecology and Evolution, Birds Canada}{Living Data Internship}{}{2022}{\begin{cvitems}
\item Data archiving for the Piping Plovers conservation pjoject
\end{cvitems}}
    \cventry{University of Bergen}{UiB Internship}{}{2020}{\begin{cvitems}
\item Website development for the Plants Functional Courses website. This included content creation as well as some front end development
\end{cvitems}}
    \cventry{University of Pretoria}{3rd year Undergraduate Mentorship Program}{}{2016}{\begin{cvitems}
\item Worked as an assistant within the M. Robertson lab. This included the sorting and identification of pitfall trap samples as well as extracting information from databases
\end{cvitems}}
\end{cventries}

\section{Funding and Awards}\label{funding-and-awards}

\begin{cventries}
    \cventry{Awarded by: British Ecological Society}{Robert May Prize}{}{2022}{}\vspace{-4.0mm}
    \cventry{Awarded by: University of Pretoria}{Qualified for the UP Postgraduate Masters Research Bursary}{}{2018}{}\vspace{-4.0mm}
    \cventry{Awarded by: University of Pretoria}{Awarded the 3rd year Undergraduate Mentorship Bursary}{}{2016}{}\vspace{-4.0mm}
\end{cventries}

\section{Working Groups and International
Collaboration}\label{working-groups-and-international-collaboration}

\begin{cventries}
    \cventry{PIs: G. Dansereau, F. Banville, M. Catchen, and T. Strydom}{Black Holes and Revelations: Identifying Priority Sampling Locations for Local Food Webs in Canada}{}{2022}{}\vspace{-4.0mm}
    \cventry{PIs: M. Liebold, P. Peres-Neto, and E. Thebault }{Merging Statistical Theory and Analyses at the Interface of Microbial and ‘Macrobial’ Ecology}{}{2022}{}\vspace{-4.0mm}
    \cventry{PIs: T. Poisot and L.J. Pollock}{Canadian metaweb construction working group}{}{2021}{}\vspace{-4.0mm}
    \cventry{PIs: T. Poisot and L.J. Pollock}{Network prediction synthesis working group}{}{2020}{}\vspace{-4.0mm}
    \cventry{PIs: V. Vandvik and B.J. Enquist}{Plant Functional Trait Course 5 in Peru}{}{2020}{}\vspace{-4.0mm}
\end{cventries}

\section{Publications}\label{publications}

\textbf{\textsuperscript{*}} Indicates co-lead author

\textsuperscript{\clubsuit} Winner of the 2022 Robert May Prize

\vspace{\baselineskip}

\subsection{\texorpdfstring{\textbf{Publications}}{Publications}}\label{publications-1}

\phantomsection\label{refs-3c1f7a2e11cecbe37c73cb7055ad6757}
\begin{CSLReferences}{1}{0}
\bibitem[\citeproctext]{ref-Christiansen2023}
1. Christiansen, D. M., \textbf{Strydom, T.}, Greiser, C., McClory, R.,
Ehrlén, J., \& Hylander, K. (2023). Effects of past and present
microclimates on northern and southern plant species in a managed forest
landscape. \emph{Accepted at Journal of Vegetation Science}.
\url{https://doi.org/10.1111/jvs.13197}

\bibitem[\citeproctext]{ref-MaitnerEstSha2021}
2. Maitner, B. S., Halbritter, A. H., Telford, R. J.,
\textbf{Strydom, T.}, Chacón-Labella, J., Henderson, A. N., Lamanna, C.,
Sloat, L. L., Kerkhoff, A. J., Messier, J., Rasmussen, N. L., Pomati,
F., Merz, E., Vandvik, V., \& Enquist, B. J. (2023). Bootstrapping
outperforms community-weighted approaches for estimating the shapes of
phenotypic distributions. \emph{Accepted at Methods in Ecology and
Evolution}. \url{https://doi.org/10.22541/au.162196147.76797968/v1}

\bibitem[\citeproctext]{ref-StrydomGraEmb2023}
3. \textbf{Strydom, T.}, Bouskila, S., Banville, F., Barros, C., Caron,
D., Farrell, M. J., Fortin, M.-J., Hemming, V., Mercier, B., Pollock, L.
J., Runghen, R., Dalla Riva, G. V., \& Poisot, T. (2023). Graph
embedding and transfer learning can help predict potential species
interaction networks despite data limitations. \emph{Methods in Ecology
and Evolution}. \url{https://doi.org/10.1111/2041-210X.14228}

\bibitem[\citeproctext]{ref-StrydomSpaEdg2022}
4. \textbf{Strydom, T.}, \& Poisot, T. (2023). SpatialBoundaries.jl:
Edge detection using spatial wombling. \emph{Ecography}.
\url{https://doi.org/10.1111/ecog.06609}

\bibitem[\citeproctext]{ref-Long-termMJRK}
5. Raath-Krüger, M. J., Schöb, C., McGeoch, M. A., Burger, D. A.,
\textbf{Strydom, T.}, \& le Roux, P. C. (2022). Long-term
spatially-replicated data show no cost to a benefactor species in a
facilitative plant-plant interaction. \emph{Oikos}.
\url{https://doi.org/10.1111/oik.09617}

\bibitem[\citeproctext]{ref-StrydomFooWeb2022}
6. \textbf{Strydom*, T.}, Bouskila*, S., Banville, F., Barros, C.,
Caron, D., Farrell, M. J., Fortin, M.-J., Hemming, V., Mercier, B.,
Pollock, L. J., Runghen, R., Dalla Riva, G. V., \& Poisot, T. (2022).
Food web reconstruction through phylogenetic transfer of low-rank
network representation. \emph{Methods in Ecology and Evolution}.
\url{https://doi.org/10.1111/2041-210X.13835}

\bibitem[\citeproctext]{ref-Chac_2020}
7. Chacón-Labella, J., Boakye, M., Enquist, B. J., Farfan-Rios, W., Gya,
R., Halbritter, A. H., Middleton, S. L., von Oppen, J.,
Pastor-Ploskonka, S., \textbf{Strydom, T.}, Vandvik, V., \& Geange, S.
R. (2021). From a crisis to an opportunity: Eight insights for doing
science in the COVID-19 era and beyond. \emph{Ecology and Evolution:
Academic Practice in Ecology and Evolution}, \emph{11}(8), 3588--3596.
\url{https://doi.org/10.1002/ece3.7026}

\bibitem[\citeproctext]{ref-Geange_2020}
8. Geange*, S. R., von Oppen*, J., \textbf{Strydom*, T.}, Boakye, M.,
Gauthier, T.-L. J., Gya, R., Halbritter, A. H., Jessup, L. H.,
Middleton, S. L., Navarro, J., Pierfederici, M. E., Chacón-Labella, J.,
Cotner, S., Farfan-Rios, W., Maitner, B. S., Michaletz, S. T., Telford,
R. J., Enquist, B. J., \& Vandvik, V. (2021). Next-generation field
courses: Integrating Open Science and online learning. \emph{Ecology and
Evolution: Academic Practice in Ecology and Evolution}, \emph{11}(8),
3577--3587. \url{https://doi.org/10.1002/ece3.7009}

\bibitem[\citeproctext]{ref-StrydomSVDEnt2020}
9. \textbf{Strydom, T.}, Dalla Riva, G. V., \& Poisot, T. (2021). SVD
entropy reveals the high complexity of ecological networks.
\emph{Frontiers in Ecology and Evolution}, \emph{9}.
\url{https://doi.org/10.3389/fevo.2021.623141}

\bibitem[\citeproctext]{ref-StrydomPriPre2021}
10. \textbf{Strydom*, T.}, Catchen*, M. D., Banville, F., Caron, D.,
Dansereau, G., Desjardins-Proulx, P., Forero-Muñoz, N. R., Higino, G.,
Mercier, B., Gonzalez, A., Gravel, D., Pollock, L. J., \& Poisot, T.
(2021). A roadmap toward predicting species interaction networks (across
space and time). \emph{Philosophical Transactions of the Royal Society
B}, \emph{376}(20210063). \url{https://doi.org/10.1098/rstb.2021.0063}

\bibitem[\citeproctext]{ref-Kattge_2020}
11. Kattge, J., Boenisch, G., Diaz, S., Lavorel, S., Prentice, C.,
Leadley, P., Wirth, C., \& the TRY Consortium. (2020). TRY plant trait
database--enhanced coverage and open access. \emph{Global Change
Biology}, \emph{26}(1), 119--188.
\url{https://doi.org/10.1111/gcb.14904}

\end{CSLReferences}

\vspace{\baselineskip}

\subsection{\texorpdfstring{\textbf{Preprints}}{Preprints}}\label{preprints}

\phantomsection\label{refs-a376e15644d103534d7467d4366343b9}
\begin{CSLReferences}{1}{0}
\bibitem[\citeproctext]{ref-HalbritterPlaTra2023}
1. Halbritter, A. H., Vandvik, V., Cotner, S., Farfan-Rios, W., Maitner,
B. S., Michaletz, S. T., Menor, I. O., Telford, R. J., Ccahuana, A.,
Cruz, R., Bravo, J. S., Andrade, P. E. S., Bustamante, L. L. V.,
Castorena, M., Chacon-Labella, J., Christiansen, C. T., Duran, S. M.,
Egelkraut, D. D., Gya, R., \ldots{} Enquist, B. J. (2023). \href{}{Plant
trait and vegetation data along a 1314 m elevation gradient with fire
history in Puna grasslands, Perú.} \emph{Under Review at Scientific
Data}.

\bibitem[\citeproctext]{ref-HiginoDesCol2023}
2. Higino, G. T., Anujan, K., Boakye, M., Degano, M. E., Forero-Muñoz,
N.-R., \& \textbf{Strydom, T.} (2023). Designing a collective prototype
of future (sub)tropical science. \emph{Preprint}.
\url{https://doi.org/10.32942/X2VC86}

\end{CSLReferences}

\section{Presentations}\label{presentations}

\vspace{\baselineskip}

\subsection{\texorpdfstring{\textbf{Invited
talks}}{Invited talks}}\label{invited-talks}

\begin{cventries}
    \cventry{\textbf{Tanya Strydom}}{What's [complexity] got to do with it?}{NetSci 2023}{Jul., 2023}{}\vspace{-4.0mm}
    \cventry{\textbf{Tanya Strydom}}{Making something out of nothing at all: Transfer learning for network prediction}{ML4MS mini-conference}{Apr., 2022}{}\vspace{-4.0mm}
    \cventry{\textbf{Tanya Strydom} alongside Aud H. Halbritter, 109 PFTC Participants}{Taking FAIR and open science to the field: The evolution of the PFTC field course}{Living Norway Colloquium}{Oct., 2020}{}\vspace{-4.0mm}
\end{cventries}
\vspace{\baselineskip}

\subsection{\texorpdfstring{\textbf{Talks}}{Talks}}\label{talks}

\begin{cventries}
    \cventry{\textbf{Tanya Strydom}, Giulio V. Dalla Riva and Timothée Poisot}{Exploring the complexity of ecological networks using SVD entropy}{11th Annual QCBS Symposium}{Dec., 2020}{}\vspace{-4.0mm}
\end{cventries}

\vspace{\baselineskip}

\subsection{\texorpdfstring{\textbf{Short Presentations and
Posters}}{Short Presentations and Posters}}\label{short-presentations-and-posters}

\begin{cventries}
    \cventry{\textbf{Tanya Strydom}, Salomé Bouskila and Timothée Poisot}{Reconstructing food webs using transfer learning}{CSEE-SCEE Annual Meeting}{Aug., 2021}{}\vspace{-4.0mm}
    \cventry{\textbf{Tanya Strydom}, Salomé Bouskila and Timothée Poisot}{Food web reconstruction using transfer learning}{12th Annual QCBS Symposium}{Dec., 2021}{}\vspace{-4.0mm}
\end{cventries}

\vspace{\baselineskip}

\subsection{\texorpdfstring{\textbf{Workshops and Organised
Sessions}}{Workshops and Organised Sessions}}\label{workshops-and-organised-sessions}

\begin{cventries}
    \cventry{Francis Banville, Gabriel Dansereau, \textbf{Tanya Strydom}}{Space Oddity: Thinking About Ecological Networks Across Space}{ESA/CSEE Meeting}{Aug., 2022}{}\vspace{-4.0mm}
    \cventry{Gracielle Higino, Mickey Boakye, Norma Forero, \textbf{Tanya Strydom}}{Designing a collective prototype of future tropical and subtropical science}{ATBC Annual Meeting}{Jul., 2021}{}\vspace{-4.0mm}
\end{cventries}

\newpage

\section{Community Engagement}\label{community-engagement}

\vspace{\baselineskip}

\subsection{\texorpdfstring{\textbf{Science
Communication}}{Science Communication}}\label{science-communication}

\begin{cventries}
    \cventry{Fortnightly cartoonist for Ecology for the Masses blog}{Cartoonist}{}{2020 - 2022}{}\vspace{-4.0mm}
\end{cventries}
\vspace{\baselineskip}

\subsection{\texorpdfstring{\textbf{Popular
Articles}}{Popular Articles}}\label{popular-articles}

\phantomsection\label{refs-83b26223daf8352e213dd849388eea17}
\begin{CSLReferences}{1}{0}
\bibitem[\citeproctext]{ref-NextGen2021}
1. von Oppen, J., Gya, R., Geange, S., \textbf{Strydom, T.}, Middleton,
S., \& Maitner, B. S. (2021). Next generation field courses: Enhancing
ECR development through open science and online learning. In
\emph{Ecology for the Masses}.

\bibitem[\citeproctext]{ref-peru2020}
2. Cotner, S., Enquist, B. J., Chacon, J., Maitner, B. S., Farfan-Rios,
W., Michaletz, S., Garen, J., Gauthier, T.-L. J., Vandvik, V., Gya, R.,
Halbritter, A. H., Hošková, K., Pierfederici, M. E.,
Quinteros-Casaverde, N. L., Diaz, E. S., Jessup, L. H.,
\textbf{Strydom, T.}, \& von Oppen, J. (2020). International scientists
need better support during global emergencies. In \emph{Times Higher
Education}.

\end{CSLReferences}



\end{document}
