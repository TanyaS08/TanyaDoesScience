%!TEX TS-program = xelatex
%!TEX encoding = UTF-8 Unicode
% Awesome CV LaTeX Template for CV/Resume
%
% This template has been downloaded from:
% https://github.com/posquit0/Awesome-CV
%
% Author:
% Claud D. Park <posquit0.bj@gmail.com>
% http://www.posquit0.com
%
%
% Adapted to be an Rmarkdown template by Mitchell O'Hara-Wild
% 23 November 2018
%
% Template license:
% CC BY-SA 4.0 (https://creativecommons.org/licenses/by-sa/4.0/)
%
%-------------------------------------------------------------------------------
% CONFIGURATIONS
%-------------------------------------------------------------------------------
% A4 paper size by default, use 'letterpaper' for US letter
\documentclass[11pt, a4paper]{awesome-cv}

% Configure page margins with geometry
\geometry{left=1.4cm, top=.8cm, right=1.4cm, bottom=1.8cm, footskip=.5cm}

% Specify the location of the included fonts
\fontdir[fonts/]

% Color for highlights
% Awesome Colors: awesome-emerald, awesome-skyblue, awesome-red, awesome-pink, awesome-orange
%                 awesome-nephritis, awesome-concrete, awesome-darknight

\definecolor{awesome}{HTML}{529FA8}

% Colors for text
% Uncomment if you would like to specify your own color
% \definecolor{darktext}{HTML}{414141}
% \definecolor{text}{HTML}{333333}
% \definecolor{graytext}{HTML}{5D5D5D}
% \definecolor{lighttext}{HTML}{999999}

% Set false if you don't want to highlight section with awesome color
\setbool{acvSectionColorHighlight}{true}

% If you would like to change the social information separator from a pipe (|) to something else
\renewcommand{\acvHeaderSocialSep}{\quad\textbar\quad}

\def\endfirstpage{\newpage}

%-------------------------------------------------------------------------------
%	PERSONAL INFORMATION
%	Comment any of the lines below if they are not required
%-------------------------------------------------------------------------------
% Available options: circle|rectangle,edge/noedge,left/right

\photo{LogoNetwork.png}
\name{Tanya}{Strydom}

\position{PhD Student}
\address{Département de sciences biologiques, Université de Montréal, Pavillon Marie-Victorin, PO Box 6128 Centre-ville Stn, Montréal, QC, H3C 3J7 Canada}

\email{\href{mailto:tanya.strydom@umontreal.ca}{\nolinkurl{tanya.strydom@umontreal.ca}}}
\homepage{tanyadoesscience.com}
\orcid{0000-0001-6067-1349}
\github{TanyaS08}
\twitter{TanyaS\_08}

% \gitlab{gitlab-id}
% \stackoverflow{SO-id}{SO-name}
% \skype{skype-id}
% \reddit{reddit-id}


\usepackage{booktabs}

% Templates for detailed entries
% Arguments: what when with where why
\usepackage{etoolbox}
\def\detaileditem#1#2#3#4#5{%
\cventry{#1}{#3}{#4}{#2}{\ifx#5\empty\else{\begin{cvitems}#5\end{cvitems}}\fi}\ifx#5\empty{\vspace{-4.0mm}}\else\fi}
\def\detailedsection#1{\begin{cventries}#1\end{cventries}}

% Templates for brief entries
% Arguments: what when with
\def\briefitem#1#2#3{\cvhonor{}{#1}{#3}{#2}}
\def\briefsection#1{\begin{cvhonors}#1\end{cvhonors}}

\providecommand{\tightlist}{%
	\setlength{\itemsep}{0pt}\setlength{\parskip}{0pt}}

%------------------------------------------------------------------------------


\usepackage{fontawesome5}


\begin{document}

% Print the header with above personal informations
% Give optional argument to change alignment(C: center, L: left, R: right)
\makecvheader

% Print the footer with 3 arguments(<left>, <center>, <right>)
% Leave any of these blank if they are not needed
% 2019-02-14 Chris Umphlett - add flexibility to the document name in footer, rather than have it be static Curriculum Vitae
\makecvfooter
  {June 2021}
    {Tanya Strydom~~~·~~~Curriculum Vitae}
  {\thepage}


%-------------------------------------------------------------------------------
%	CV/RESUME CONTENT
%	Each section is imported separately, open each file in turn to modify content
%------------------------------------------------------------------------------



\hypertarget{research-interests}{%
\section{Research Interests}\label{research-interests}}

computational ecology; functional traits; ecological networks; species interactions; FAIR and open science

\hypertarget{education}{%
\section{Education}\label{education}}

\detailedsection{\detaileditem{Université de Montréal}{2020 - Present}{Doctor of Philosophy: Biological Sciences}{Montréal, Canada}{\item{Advisor: T. Poisot, PhD}\item{Thesis: Common properties of species interaction networks across space}}\detaileditem{Stockholms Universitet}{2018-20}{Master of Science: Ecology and Biodiversity}{Stockholm, Sweden}{\item{Advisor: K. Hylander, PhD}\item{Thesis: Declines and increases in northern and southern plant populations after changes in the microclimate}}\detaileditem{University of Pretoria}{2017}{Bachelor of Science (Honours): Plant Sciences}{Pretoria, South Africa}{\item{Advisor: P.C. le Roux, PhD}\item{Thesis: Bush encroachment in South Africa’s montane grasslands: the impact of \textit{Leucosidea sericea} on microclimate and vegetation}}\detaileditem{University of Pretoria}{2014-16}{Bachelor of Science: Ecology}{Pretoria, South Africa}{\empty}}

\hypertarget{research-experience}{%
\section{Research Experience}\label{research-experience}}

\detailedsection{\detaileditem{V. Vandvik, PhD and B.J. Enquist, PhD}{2020}{Plant functional trait responses to elevation and fire}{}{\item{Attended the 5th Plants Functional Traits Course in Peru co-hosted by the Univesity of Bergen and Arizona University.}\item{Theory on plant functional traits and their relationships with broader ecological processes.}\item{Practical elements include: experimental design, data collection and curation, and report writing.}}\detaileditem{K. Hylander, PhD}{2019-20}{Plant vital rates along microclimate gradients.}{}{\item{Masters degree research project.}\item{Focused on the variation of plant population vital rates along microclimate gradients and the role of historic climatic conditions.}}\detaileditem{J. Ehrlén, PhD and A. Tack, PhD}{2019}{Plant-pollinator interactions in different microsites}{}{\item{A Masters level course research project.}\item{Independently worked on hypothesis formulation and experimental design.}}\detaileditem{P.C. le Roux, PhD}{2017}{The impact of an encroaching species on vegetation and microclimate}{}{\item{Honours degree research project.}\item{Research focused on the concepts of biotic interactions, ecosystem engineering and habitat modification.}}}

\newpage

\hypertarget{academic-service}{%
\section{Academic Service}\label{academic-service}}

\detailedsection{\detaileditem{Student Member}{2020 - Present}{Québec Centre for Biodiversity Sciences}{}{\empty}\detaileditem{Plant Ecology, PLoS ONE}{}{Reviewed Journals}{}{\empty}}

\hypertarget{publications}{%
\section{Publications}\label{publications}}

\vspace{\baselineskip}

\hypertarget{publications-1}{%
\subsection{\texorpdfstring{\textbf{Publications}}{Publications}}\label{publications-1}}

\begingroup
\setlength{\parindent}{-0.5in}
\setlength{\leftskip}{0.5in}

\hypertarget{refs_journals}{}
\leavevmode\hypertarget{ref-Chac_2020}{}%
Chacón-Labella, J., Boakye, M., Enquist, B. J., Farfan-Rios, W., Gya, R., Halbritter, A. H., Middleton, S. L., von Oppen, J., Pastor-Ploskonka, S., \textbf{Strydom, T.}, Vandvik, V., \& Geange, S. R. (2021). From a crisis to an opportunity: {Eight} insights for doing science in the {COVID}-19 era and beyond. \emph{Ecology and Evolution}, \emph{11}(8), 3588--3596. \url{https://doi.org/10.1002/ece3.7026}

\leavevmode\hypertarget{ref-Geange_2020}{}%
Geange, S. R., von Oppen, J., \textbf{Strydom, T.}, Boakye, M., Gauthier, T.-L. J., Gya, R., Halbritter, A. H., Jessup, L. H., Middleton, S. L., Navarro, J., Pierfederici, M. E., Chacón-Labella, J., Cotner, S., Farfan-Rios, W., Maitner, B. S., Michaletz, S. T., Telford, R. J., Enquist, B. J., \& Vandvik, V. (2021). Next-generation field courses: {Integrating Open Science} and online learning. \emph{Ecology and Evolution}, \emph{11}(8), 3577--3587. \url{https://doi.org/10.1002/ece3.7009}

\leavevmode\hypertarget{ref-Kattge_2020}{}%
Kattge, J., Boenisch, G., Diaz, S., Lavorel, S., Prentice, C., Leadley, P., Wirth, C., \& the TRY Consortium. (2020). \emph{Global Change Biology}, \emph{26}(1), 119--188. \url{https://doi.org/10.1111/gcb.14904}

\endgroup
\vspace{\baselineskip}

\hypertarget{preprints}{%
\subsection{\texorpdfstring{\textbf{Preprints}}{Preprints}}\label{preprints}}

\begingroup
\setlength{\parindent}{-0.5in}
\setlength{\leftskip}{0.5in}

\hypertarget{refs_review}{}
\leavevmode\hypertarget{ref-MaitnerEstSha2021}{}%
Maitner, B. S., Halbritter, A. H., Telford, R. J., \textbf{Strydom, T.}, Chacón-Labella, J., Henderson, A. N., Lamanna, C., Sloat, L. L., Kerkhoff, A. J., Messier, J., Rasmussen, N. L., Pomati, F., Merz, E., Vandvik, V., \& Enquist, B. J. (2021). On estimating the shape and dynamics of phenotypic distributions in ecology and evolution. \emph{Submitted}. \url{https://doi.org/10.22541/au.162196147.76797968/v1}

\leavevmode\hypertarget{ref-StrydomPriPre2021}{}%
\textbf{Strydom, T.}, Catchen, M. D., Banville, F., Caron, D., Dansereau, G., Desjardins-Proulx, P., Forero-Muñoz, N. R., Higino, G., Mercier, B., Gonzalez, A., Gravel, D., Pollock, L. S., \& Poisot, T. (2021). A primer on predicting species interaction networks (across space and time). \emph{Submitted}. \url{https://doi.org/10.32942/osf.io/eu7k3}

\leavevmode\hypertarget{ref-StrydomSVDEnt2020}{}%
\textbf{Strydom, T.}, Dalla Riva, G. V., \& Poisot, T. (2021). SVD entropy reveals the high complexity of ecological networks. \emph{Accepted at Frontiers in Ecology and Evolution}. \url{https://doi.org/10.32942/osf.io/q9v85}

\endgroup
\vspace{\baselineskip}

\hypertarget{science-communication-engagement}{%
\section{Science Communication \& Engagement}\label{science-communication-engagement}}

\vspace{\baselineskip}

\hypertarget{communication}{%
\subsection{\texorpdfstring{\textbf{Communication}}{Communication}}\label{communication}}

\detailedsection{\detaileditem{Fortnightly cartoonist for Ecology for the Masses blog}{2020 - present}{Cartoonist}{}{\empty}}
\vspace{\baselineskip}

\hypertarget{popular-articles}{%
\subsection{\texorpdfstring{\textbf{Popular Articles}}{Popular Articles}}\label{popular-articles}}

\begingroup
\setlength{\parindent}{-0.5in}
\setlength{\leftskip}{0.5in}

\hypertarget{refs_popular}{}
\leavevmode\hypertarget{ref-NextGen2021}{}%
von Oppen, J., Gya, R., Geange, S., \textbf{Strydom, T.}, Middleton, S., \& Maitner, B. S. (2021). Next generation field courses: Enhancing ECR development through open science and online learning. In \emph{\href{https://ecologyforthemasses.com/2021/03/08/next-generation-field-courses-enhancing-ecr-development-through-open-science-and-online-learning/}{Ecology for the Masses}}.

\leavevmode\hypertarget{ref-peru2020}{}%
Cotner, S., Enquist, B. J., Chacon, J., Maitner, B. S., Farfan-Rios, W., Michaletz, S., Garen, J., Gauthier, T.-L. J., Vandvik, V., Gya, R., Halbritter, A. H., Hošková, K., Pierfederici, M. E., Quinteros-Casaverde, N. L., Diaz, E. S., Jessup, L. H., \textbf{Strydom, T.}, \& von Oppen, J. (2020). International scientists need better support during global emergencies. In \emph{\href{https://tinyurl.com/y5ccw9nb}{Times Higher Education}}.

\endgroup

\newpage

\hypertarget{presentations}{%
\section{Presentations}\label{presentations}}

\vspace{\baselineskip}

\hypertarget{invited-talks}{%
\subsection{\texorpdfstring{\textbf{Invited talks}}{Invited talks}}\label{invited-talks}}

\detailedsection{\detaileditem{Living Norway Colloquium 2020: Towards openess and transparency in applied ecology}{Oct., 2020}{Taking FAIR and open science to the field: The evolution of the PFTC field course}{Trondheim, Norway}{\item{Presented as part of the education and open science workshop}\item{\textbf{Tanya Strydom} alongside Aud H. Halbritter, 109 PFTC Participants}\item{\href{https:doi.org/10.5281/zenodo.4117503}{Link to slides}}}}
\vspace{\baselineskip}

\hypertarget{talks}{%
\subsection{\texorpdfstring{\textbf{Talks}}{Talks}}\label{talks}}

\detailedsection{\detaileditem{11th Annual QCBS Symposium}{Dec., 2020}{Exploring the complexity of ecological networks using SVD entropy}{Québec, Canada}{\item{\textbf{Tanya Strydom}, Giulio V. Dalla Riva and Timothée Poisot}\item{\href{https://tanyas08.github.io/Talks/2020_Dec_QCBS/index.html}{Link to slides}}}}

\hypertarget{technical-skills}{%
\section{Technical Skills}\label{technical-skills}}

\begin{cvskills}
  \cvskill
    {Statistical Analysis}
    {generalized and linear mixed-effect models; mulitvariate analysis; Bayesian analysis; primarily using R}

  \cvskill
    {Spatial Analysis}
    {spatial analysis in ecology; ArcGIS; Maxent; GBIF}

  \cvskill
    {Image Analysis}
    {ImageJ; Adobe Photoshop}

  \cvskill
    {Phylogenetic Analysis}
    {extracting and cleaning samples from GenBank; MEGA}

  \cvskill
    {Academic Writing}
    {assessed at various levels; peer-reviewed articles; literature reviews; research proposals; reports; popular articles}

  \cvskill
    {Oral Communication}
    {masters level course; presented in various settings}

  \cvskill
    {Language Skills}
    {English and Afrikaans as a native speaker; conversational in German}

\end{cvskills}

\hypertarget{internships}{%
\section{Internships}\label{internships}}

\detailedsection{\detaileditem{University of Bergen}{2020}{UiB Internship}{}{\item{Website development for the Plants Functional Courses website. This included content creation as well as some front end development}}\detaileditem{University of Pretoria}{2016}{3rd year Undergraduate Mentorship Program}{}{\item{Worked as an assistant within the M. Robertson lab. This included the sorting and identification of pitfall trap samples as well as extracting information from databases}}}

\hypertarget{funding-and-awards}{%
\section{Funding and Awards}\label{funding-and-awards}}

\detailedsection{\detaileditem{Awarded by: University of Pretoria}{2018}{Qualified for the UP Postgraduate Masters Research Bursary}{}{\empty}\detaileditem{Awarded by: University of Pretoria}{2016}{Awarded the 3rd year Undergraduate Mentorship Bursary}{}{\empty}}

\end{document}
